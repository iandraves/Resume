\documentclass{resume} % Use the custom resume.cls style

\usepackage[left=0.4 in,top=0.4in,right=0.4 in,bottom=0.4in]{geometry} % Document margins
\newcommand{\tab}[1]{\hspace{.2667\textwidth}\rlap{#1}} 
\newcommand{\itab}[1]{\hspace{0em}\rlap{#1}}
\name{Ian Draves}
\address{(970) 817-4249 \\ Fort Collins, CO}
\address{\href{mailto:ian.draves@colorado.edu}{ian.draves@colorado.edu} \\ \href{https://www.linkedin.com/in/iandraves}{linkedin.com/in/iandraves} \\ \href{https://github.com/iandraves}{github.com/iandraves}}  %

\begin{document}

%----------------------------------------------------------------------------------------
%----------------------------------------------------------------------------------------
%	EDUCATION SECTION
%----------------------------------------------------------------------------------------
\begin{rSection}{Education}
    {\bf Bachelor of Arts in Computer Science}, University of Colorado Boulder \hfill {May 2025}\\
    GPA: 3.95 | Relevant courses: Data Structures, Computer Systems, Discrete Math
\end{rSection}

%----------------------------------------------------------------------------------------
% Experience
%----------------------------------------------------------------------------------------
\begin{rSection}{EXPERIENCE}
    \textbf{Software Developer} \hfill Feb 2022 | Present\\
    Blueprint Boulder \hfill \textit{Boulder, CO}
    \begin{itemize}
    \itemsep -3pt {} 
        \item Used React Native to successfully build and deploy a hotline app for the Colorado Immigrant Rights Coalition. 
        \item Added multi-language support with i18n library, allowing for quick translation between English and Spanish.
        \item Implemented the Public Resource page from a Figma wire-frame, creating various custom React components.
    \end{itemize}
\end{rSection}

%----------------------------------------------------------------------------------------
% Projects
%----------------------------------------------------------------------------------------
\begin{rSection}{PROJECTS}
    \vspace{-1.25em}
    \item \textbf{ML Stock Predictor} {(Python, finnhub API, Yahoo Finance API, pandas, numpy, h2o)} \hfill \href{https://github.com/iandraves/InsiderAI}{GitHub}
    \begin{itemize}
        \itemsep -3pt {} 
        \item Used publicly available SEC Form 4 insider trading data to train a GBM model to predict stock movements.
        \item Collected over 10 months of insider trading and stock price data with the finnhub and Yahoo Finance APIs.
        \item Cleaned data using pandas and numpy, feeding the resulting dataset into multiple models to forecast prices.
        \item Successfully achieved prediction confidence rates of over 95\%, suggesting a strong correlation between insider trades and future prices. 
    \end{itemize}
    \item \textbf{Social Media Dashboard} {(React, Sortable.js)} \hfill \href{https://github.com/iandraves/Dashli}{GitHub}
    \begin{itemize}
        \itemsep -3pt {} 
        \item Created a responsive social media dashboard, using React and vanilla CSS for dynamic and styled components.
        \item Implemented light and dark color themes, using localStorage to make the selected color theme persistent.
        \item Made dashboard interactive with Sortable.js for draggable components and CSS animations for two-sided cards.
    \end{itemize}
    \item \textbf{Note Taker for Debate} {(JavaScript, HTML, CSS, UIKit)} \hfill \href{https://github.com/iandraves/Droplet}{GitHub}
    \begin{itemize}
        \itemsep -3pt {} 
        \item Created speed focused note taker for competitive debate using vanilla JavaScript. 
        \item Used UIKit for modern, minimal, and focused component design and web-page animations.
        \item Implemented keybindings as a primary way of interacting with the web app for efficient end-user control.
    \end{itemize}
    \item \textbf{Wikipedia Degrees of Separation Calculator} {(Python, Wikipedia API)} \hfill \href{https://github.com/iandraves/WikiWeb}{GitHub}
    \begin{itemize}
        \itemsep -3pt {} 
        \item Created a script to calculate the degrees of separation between any two Wikipedia pages in real-time.
        \item Optimized real-time search with multi-directional concurrency using Python's concurrency module.
        \item Crawled and searched for Wikipedia pages using Wikipedia's Python API.
    \end{itemize}
    \item \textbf{Implicit Bias Deliberate Re-association Engine} {(JavaScript, HTML, CSS, UIKit)} \hfill \href{https://github.com/iandraves/RevisIA}{GitHub}
    \begin{itemize}
        \itemsep -3pt {} 
        \item Created a re-association web app in JavaScript inspired by Harvard's ``Weapons" Implicit Association Test.
        \item Implemented modern and responsive user interface with UIKit for an intuitive user experience.
        \item Successfully created complex interactivity without HTML5 Canvas and deployed the site to GitHub pages.
    \end{itemize}
\end{rSection} 

%----------------------------------------------------------------------------------------
% SKILLS	
%----------------------------------------------------------------------------------------
\begin{rSection}{SKILLS}
    \begin{tabular}{ @{} >{\bfseries}l @{\hspace{6ex}} l }
        Languages & JavaScript (proficient), Python (proficient), C++ (proficient), Go (familiar), Rust (familiar) \\ \\
        Tech/Concepts & React, Node.js, Bootstrap, UIKit, Linux, Git, GitHub, Data Structures, Algorithms, OOP
    \end{tabular}\\
    \end{rSection}

%----------------------------------------------------------------------------------------

\end{document}