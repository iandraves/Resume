\documentclass{resume} % Use the custom resume.cls style

\usepackage[left=0.4 in,top=0.4in,right=0.4 in,bottom=0.4in]{geometry} % Document margins
\newcommand{\tab}[1]{\hspace{.2667\textwidth}\rlap{#1}} 
\newcommand{\itab}[1]{\hspace{0em}\rlap{#1}}
\name{Ian Draves}
\address{(970) 817-4249 \\ Boulder, CO}
\address{\href{mailto:ian.draves@colorado.edu}{ian.draves@colorado.edu} \\ \href{https://www.linkedin.com/in/iandraves}{linkedin.com/in/iandraves} \\ \href{https://github.com/iandraves}{github.com/iandraves}}  %

\begin{document}

%----------------------------------------------------------------------------------------
%----------------------------------------------------------------------------------------
%	EDUCATION SECTION
%----------------------------------------------------------------------------------------
\begin{rSection}{Education}
    {\bf Bachelor of Arts in Computer Science}, University of Colorado Boulder \hfill {May 2025}\\
    GPA: 3.95 | Relevant courses: Data Structures, Computer Systems, Discrete Math
\end{rSection}

%----------------------------------------------------------------------------------------
% Experience
%----------------------------------------------------------------------------------------
\begin{rSection}{EXPERIENCE}
    \textbf{Software Developer} \hfill Feb 2022 | Present\\
    Blueprint Boulder \hfill \textit{Boulder, CO}
    \begin{itemize}
    \itemsep -3pt {} 
        \item Used React Native to successfuly build and deploy a hotline app for the Colorado Immigrant Rights Coalition. 
        \item Added multi-language support with i18n library, allowing for quick translation between English and Spanish.
        \item Implemented the Public Resource page from a Figma wireframe, creating various custom React components.
    \end{itemize}
\end{rSection}

%----------------------------------------------------------------------------------------
% Projects
%----------------------------------------------------------------------------------------
\begin{rSection}{PROJECTS}
    \vspace{-1.25em}
    \item \textbf{Social Media Dashboard} {(React, Sortable.js)} \hfill \href{https://github.com/iandraves/SocialMediaDashboard}{GitHub}
    \begin{itemize}
        \itemsep -3pt {} 
        \item Created a responsive social media dashboard, using React and vanilla CSS for dynamic and styled components.
        \item Implemented light and dark color themes, using localStorage to make the selected color theme persistent.
        \item Made dashboard interactive with Sortable.js for draggable components and CSS animations for two-sided cards.
    \end{itemize}
    \item \textbf{Note Taker for Debate} {(JavaScript, HTML, CSS, UIKit)} \hfill \href{https://github.com/iandraves/Droplet}{GitHub}
    \begin{itemize}
        \itemsep -3pt {} 
        \item Created speed focused note taker for competitive debate using vanilla JavaScript. 
        \item Used UIKit for modern, minimal, and focused component design and web-page animations.
        \item Implemented keybindings as a primary way of interacting with the web app for efficient end-user control.
    \end{itemize}
    \item \textbf{Wikipedia Degrees of Separation Calculator} {(Python, Wikipedia API)} \hfill \href{https://github.com/iandraves/WikiWeb}{GitHub}
    \begin{itemize}
        \itemsep -3pt {} 
        \item Created a script to calculate the degrees of separation between any two Wikipedia pages in real-time.
        \item Optimized real-time search with multi-directional concurrency using Python's concurrency module.
        \item Crawled and searched for Wikipedia pages using Wikipedia's Python API.
    \end{itemize}
    \item \textbf{Implicit Bias Deliberate Re-association Engine} {(JavaScript, HTML, CSS, UIKit)} \hfill \href{https://github.com/iandraves/RevisIA}{GitHub}
    \begin{itemize}
        \itemsep -3pt {} 
        \item Created a re-association web app in JavaScript inspired by Harvard's ``Weapons" Implicit Association Test.
        \item Implemented modern, responsive user interface with UIKit for an intuitive user experience.
        \item Successfully created complex interactivity without HTML5 Canvas and deployed site to GitHub pages.
    \end{itemize}
    \item \textbf{Evidence Collection Script for Debate} {(Python, BeautifulSoup)} \hfill \href{https://github.com/iandraves/autodocs}{GitHub}
    \begin{itemize}
        \itemsep -3pt {} 
        \item Created Python script to automate the task of downloading evidence for debate from OpenEvidence.
        \item Scraped the OpenEvidence website with BeautifulSoup to automatically find and download docx files.
        \item Optimized download rate with multi-threading using Python's threading module.
        \item Used by my debate team in the 2021 season when we made nationals for the first time in over a decade.
    \end{itemize}
\end{rSection} 

%----------------------------------------------------------------------------------------
% TECHINICAL STRENGTHS	
%----------------------------------------------------------------------------------------
\begin{rSection}{Technical Strengths}
    \begin{tabular}{ @{} >{\bfseries}l @{\hspace{6ex}} l }
        Languages & JavaScript (proficient), Python (proficient), C++ (proficient), Go (familiar), Rust (familiar) \\ \\
        Tech/Concepts & React, Node.js, Bootstrap, UIKit, Linux, Git, GitHub, Data Structures, Algorithms, OOP
    \end{tabular}\\
    \end{rSection}

%----------------------------------------------------------------------------------------

\end{document}